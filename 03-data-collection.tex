This section describes the systematic approach we used to construct the dataset for our study on behavioral incompatibilities caused by newly added unchecked exceptions in upgraded Java libraries. 

Broadly, we require three key components: Java-based clients that depend on third-party libraries, the current versions of the libraries used by these clients, and the latest available versions of those same libraries. For each client-library pair, we also need to extract the set of unchecked exceptions thrown by the library methods actually used by the client.

To obtain this information, our methodology involves several key steps: identifying suitable Java clients, extracting their library dependencies, resolving both the current and latest versions of the libraries, analyzing exception behavior in both versions, and recording all methods that introduce newly added unchecked exceptions. 

Each step is essential for enabling our tool, UnCheckGuard, to pinpoint client call sites that may be affected by behavioral breaking changes.

\subsection{Collecting Clients}

To begin our analysis, we first collected suitable client projects. We used the DUETS dataset~\cite{durieux21:_duets}, which provides a curated list of Java-based clients hosted on GitHub, each with at least five stars. We attempted to download each client repository and discarded any client that failed to download.

Next, we checked whether the project included a \texttt{pom.xml} file, which indicates that it is a Maven-based project. This step was essential, as our analysis depends on running Maven commands. We compiled each client to produce a JAR file and kept only those clients that compiled successfully for further analysis.

