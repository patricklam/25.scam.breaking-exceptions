We continue with a motivating example drawn from the DUETS collection~\cite{durieux21:_duets}
of client/library pairs. 

We start with our client, \texttt{HttpAsyncClientUtils}, which is one of the clients in DUETS. This client declares a dependency on
version 4.4.6 of the \texttt{httpcore} library\footnote{\url{https://hc.apache.org/index.html}}. Since the release of the version of \texttt{HttpAsyncClientUtils} that we are using, the \texttt{httpcore} developers
have released a number of new versions, and at the time of writing, the latest version of \texttt{httpcore}
is 4.4.16\footnote{While \texttt{httpcore} 5.2.4 is in fact the latest version of this library, the library developers have released \texttt{httpcore5} as a distinct Maven component from \texttt{httpcore4}, and labelled \texttt{httpcore}(4) as end-of-life.}.

A revision of the \texttt{httpcore} library between 4.4.6 and 4.4.16 adds a check for an
error condition.  If the condition evaluates to true, the library method will
explicitly throw an \texttt{IllegalArgumentException}. The client, \texttt{HttpAsyncClientUtils}\footnote{\url{https://github.com/iiweniiang/HttpAsyncClientUtils}},
calls the relevant part of the library, and thus may be affected by the new exception. We explain how UnCheckGuard finds this exception (and how it avoids some false positives).


\paragraph{Library} All constructors for the \texttt{org.apache.http.HttpHost} class transitively call
the static method \texttt{Args.containsNoBlanks()}. Between version 4.4.6 and version 4.4.16, the \texttt{httpcore}
developers added the following lines of code to \texttt{containsNoBlanks()}:
\begin{lstlisting}[style=javacode]
    if (argument.length() == 0) {
       throw new IllegalArgumentException
                    (name + " may not be empty");
    }
\end{lstlisting}
Specifically, all \texttt{HttpHost} constructors take a \texttt{hostname} parameter and call \texttt{containsNoBlanks()}
with that parameter (to check that it contains no blanks). It is therefore possible to trigger this newly-thrown
exception in a client by attempting to instantiate a new \texttt{HttpHost} object and passing it an empty
\texttt{hostname}.

Our UnCheckGuard tool analyzes the change in \texttt{httpcore} and finds that, in
version 4.4.16, all of the \texttt{HttpHost} constructors may now throw an
\texttt{IllegalArgumentException} via the \texttt{containsNoBlanks()} method.
This exception was not thrown in 4.4.6.

To detect this change, UnCheckGuard processes JAR files for both \texttt{httpcore-4.4.6} and \texttt{httpcore-4.4.16}. It uses SootUp~\cite{Karakaya24:_sootup} to construct a call graph using Class Hierarchy Analysis (CHA) starting from the public \texttt{<init>(String, int)}\footnote{Specifically, constructor \texttt{<init>(String, int)} returning a \texttt{void} on class \texttt{org.apache.http.HttpHost}} constructor on \texttt{HttpHost} and identifies the set of all methods transitively reachable by the client (which we will discuss below). UnCheckGuard then collects all unchecked exceptions thrown within this set of reachable methods, for both library versions.

\paragraph{Client} 
A newly-added exception is only relevant to a particular client if that client may
potentially trigger that exception. We found that, often, a client will call a library,
and the library code really does contain a newly-added exception, but there is no way for
the \emph{client} to cause the library to reach that exception.
But, in this case, it turns out that
our \texttt{HttpAsyncClientUtils} client has reachable code from its public \texttt{createAsyncClient(boolean)}\footnote{Fully-qualified: method \texttt{createAsyncClient(boolean)} returning a \texttt{CloseableHttpAsyncClient} on class \texttt{Util.HttpClientUtil.HttpAsyncClient}.} method
that creates an \texttt{HttpHost} with an empty \texttt{host}. This
method takes a \texttt{proxy}
parameter and contains the following code:
\begin{lstlisting}[style=javacode]
if (proxy) {
  return HttpAsyncClients.custom()
   .setConnectionManager(conMgr)
   .setDefaultCredentialsProvider(credentialsProvider)
   .setDefaultAuthSchemeRegistry(authSchemeRegistry)
(*@\Highlight{   .setProxy(new HttpHost(host, port))}@*)
   .setDefaultCookieStore(new BasicCookieStore())
   .setDefaultRequestConfig(requestConfig).build();
} else {
  // ...
}
\end{lstlisting}

The \texttt{HttpAsyncClientUtils} client declares the two variables (host and port) required for \texttt{HttpHost} in the following way:
\begin{lstlisting}[language=Java,basicstyle=\scriptsize\ttfamily]
    private String host = "";
    private int port = 0;
   // ...
\end{lstlisting}
where \texttt{host} is a private field initialized to the empty string.
Thus, calling \texttt{createAsyncClient(true)} triggers an exception when executed with
\texttt{httpcore} version 4.4.16 but not with 4.4.6. Conversely, one can imagine a library design
where \texttt{HttpHost} objects are always initialized with a \texttt{hostname} of \texttt{"localhost"},
such that the newly-added exception would not be triggerable.

To detect that our \texttt{HttpAsyncClientUtils} client calls a method from \texttt{httpcore-4.4.6} which, upon upgrading to \texttt{httpcore-4.4.16}, may throw a new unchecked exception, UnCheckGuard begins by identifying all external library methods invoked anywhere in the client. It then analyzes both the current and the latest versions of the library to determine whether any newly introduced unchecked exceptions are reachable from the client's code. Here, reachability means that the client can trigger the exception in the library on some execution of the program, using values it passes to the library as parameters.

To check if the client-supplied values can reach the exception-throwing site, we use FlowDroid~\cite{Arzt14:_flowdroid}'s taint analysis. Taint analysis is essential in this scenario because the existence of a control-flow path from the client callsite to an exception-throwing statement is not sufficient to conclude that the exception is actually triggerable by the client. As alluded to above, we found that many such paths may exist in a library, but the path conditions leading to the exception might depend entirely on internal library values, rather than on client-supplied inputs; it is impossible for our client to cause the execution of any path that triggers the exception. In our experience, taint analysis helps distinguish actual behavioural breaking changes from false positives.

Specifically, we use taint analysis to track whether any client-supplied method parameters to library calls (source) can propagate to the exception object's constructor (sink). If taint analysis determines that no client-supplied input flows into the exception-triggering logic, then we can conclude that the newly added exception will not cause a behavioural breaking change, and we do not report that exception.
%For exceptions that we report, we know that a path exists in the interprocedural control-flow graph (which one can compute with CHA) from the client method to the statement throwing the exception in the library.

In version 4.4.6, UnCheckGuard finds two sites throwing \texttt{IllegalArgumentException}, while in 4.4.16, it detects three—each of which the client can potentially trigger using the values it chooses to pass to the library as parameters.

Based on FlowDroid's confirmation of the reachability of the new exception's constructor, we report that the library-client pair \texttt{HttpAsyncClientUtils} and \texttt{httpcore} exhibits a behavioural breaking change.

Given this report, it is straightforward to write a test case that calls the client's \texttt{createAsyncClient()} method
and triggers the exception after an upgrade:
\newpage
\begin{lstlisting}[language=Java,basicstyle=\scriptsize\ttfamily]
@Test
void testCreateAsyncClientThrowsExceptionForEmptyProxyHost() {
  HttpAsyncClient client = new HttpAsyncClient();

  IllegalArgumentException exception =
    assertThrows(IllegalArgumentException.class, () -> {
        client.createAsyncClient(true);
    });

    assertTrue(exception.getMessage()
    .contains("may not be empty"),
      "Expected exception due to empty hostname "+
      "after upgrading to httpcore-4.4.16");
}
\end{lstlisting}
