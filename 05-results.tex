As discussed in Section~\ref{sec:data-collection}, we evaluated UnCheckGuard on 36 Java-based clients from the DUETS dataset~\cite{durieux21:_duets}.

The goal of our tool is to detect whether a client calls a library method that, upon upgrading the library to a newer version, introduces a previously non-existent unchecked exception—potentially resulting in a behavioural breaking change.

We explore the following research questions:

\begin{itemize}
  \item[\textbf{RQ1:}] How often do published changes to Java libraries throw new unchecked exceptions in methods,
and under what circumstances do such exceptions occur (e.g. major/minor/patch versions)?
  \item[\textbf{RQ2:}]  Do library clients, in practice, call methods with new added exceptions, and is it possible for the clients to trigger these exceptions? Is it possible to write client test cases that trigger the exceptions?
\end{itemize}


\begin{table}[h]
\centering
\caption{Exception Analysis Funnel}
\label{tab:exception-funnel}
\begin{tabular}{l r}
\toprule
\textbf{Stage} & \textbf{Count} \\
\midrule
Client Invocations of External Methods & 8048 \\
Newly Added Exceptions Called by Clients & 285 \\
Exceptions Passing Taint Analysis & 49 \\
Exceptions with a Test Case & 3 \\
\bottomrule
\end{tabular}
\end{table}


\subsection{Newly-added Unchecked Exceptions in Java Libraries}

Our evaluation included 36 client applications, which depended on 69 distinct libraries. Across these, we formed 99 client-library pairs, each corresponding to a combination of a specific client and one of the libraries that it depends on. Table~\ref{tab:version-changes} presents our clients and libraries, the number of client callsites invoking methods in the library with newly-added exceptions, and the number of these callsites that pass the taint analysis reachability filter.

UnCheckGuard detected 285 callsites across these 99 pairs where the upgraded version of the library could throw a new unchecked exception. However, it was not possible to trigger all of these exceptions using the client's methods, even with a free choice of parameters to pass to the client code. We therefore applied a taint-based reachability analysis to filter out cases that definitely could not result in actual runtime failures. After this filtering step, we identified 49 callsites in total—spanning 8 distinct libraries—that appeared to potentially be affected by a newly added unchecked exception.

Semantic versioning~\cite{preston-werner23:_seman_version} proposes that version numbers have three parts, $x.y.z$. According to semantic versioning, library developers are to change the major version $x$ when an upgrade is breaking---that is, a client may have to modify their code to use the new versioning. Minor version changes $y$ may include new features, while patch changes $z$ fix bugs.

<<<<<<< HEAD
Notably, Table~\ref{tab:version-distribution} showcases that 7 out of these 8 libraries introduced new unchecked exceptions as part of a major version bump. However, we also observed one case in a patch version upgrade. This indicates that even smaller upgrades may introduce behavioural breaking changes via unchecked exceptions—something developers may not anticipate.
=======
Table~\ref{tab:version-distribution} shows the distribution of newly-added exceptions reachable from clients, across upgrade types. Notably, 7 out of these 8 libraries introduced new unchecked exceptions as part of a major version bump. However, we also observed one case in a patch version upgrade. This indicates that even smaller upgrades may introduce behavioural breaking changes via unchecked exceptions—something developers may not anticipate.
>>>>>>> 3fb9cd467d9eb201e59d55fa15fb551b7823d925

\vspace{1em}
\begin{tcolorbox}[colback=gray!10, colframe=black]
\textbf{Answer RQ1:} Java libraries introduce newly added unchecked client-relevant exceptions across versions frequently enough to be relevant to clients. We found newly added unchecked exceptions in 8 out of 69 distinct libraries (11.5\%). These changes occurred in major version upgrades (7 times) but also in patch (1) version upgrades (e.g., \texttt{httpcore-4.4.6}~$\rightarrow$~\texttt{httpcore-4.4.16}).
\end{tcolorbox}
\vspace{1em}

\begin{table}[h]
\centering
\caption{Distribution of Reachable Newly-Added Exceptiosn Across Version Types}
\label{tab:version-distribution}
\begin{tabular}{lcc}
\toprule
\textbf{Version Type} & \textbf{Libraries} & \textbf{Affected Call Sites} \\
\midrule
Major Version Change & 7 & 48 \\
Minor Version Change & 0 & 0 \\
Patch Version Change & 1 & 1 \\
\bottomrule
\end{tabular}
\end{table}

\begin{table*}[hbt!]
\centering
\caption{Version Changes and Associated Clients with Occurrence Counts}
\label{tab:version-changes}
\begin{tabular}{>{\raggedright\arraybackslash\hangindent=2em}p{3.5cm} >{\raggedright\arraybackslash\hangindent=2em}p{3.5cm} >{\raggedright\arraybackslash\hangindent=2em}p{3.5cm} >{\raggedleft\arraybackslash}p{2cm} >{\raggedleft\arraybackslash}p{2cm}}
\toprule
\textbf{Client} & \textbf{Current Version} & \textbf{Latest Version} & \textbf{Number of Callsites} & \textbf{Reachable Callsites} \\
\midrule
95MISTAKE/sonar	& sonar-plugin-api-7.4	& sonar-plugin-api-9.4.0.54424 & 2 & 2 \\
4ntoine/ServiceDiscovery-java	& protobuf-java-2.6.1 &	protobuf-java-4.31.1 & 16 & 1 \\
72crm/72crm-java & poi-3.17 & poi-5.4.1 & 30 & 1 \\
269941633/spring-boot-mybatis-redis & pagehelper-4.1.6 & pagehelper-6.1.0 & 1 & \\
269941633/spring-boot-mybatis-mysql-write-read & & & 1 & \\
6ag/im-demo-netty-tcp-websocket & netty-all-4.1.32.Final & netty-all-5.0.0.Alpha2 & 6 & 1 \\
527515025/JavaTest & maxmind-db-1.2.2 & maxmind-db-3.2.0 & 1 & \\
72crm/72crm-java & jfinal-4.5 & jfinal-5.2.5 & 41 & 41 \\
266945/GOIM	& jfinal-3.8	& jfinal-5.2.2 & 2 & \\
527515025/JavaTest	& jedis-2.9.0	& jedis-6.0.0 & 1 & \\
72crm/72crm-java	& hutool-all-4.6.8	& hutool-all-5.8.38 & 3 & 1 \\
a63881763/HttpAsyncClientUtils	& httpcore-4.4.6	& httpcore-4.4.16 & 12 & 1 \\
0RaymondJiang0/tushare-java & & & 1 & \\
527515025/JavaTest & & & 12 & \\
527515025/JavaTest & httpclient-4.5.2 &	httpclient-4.5.14 & 5 & \\
a63881763/HttpAsyncClientUtils & & & 5 & \\
99soft/sameas4j	& gson-1.6 & gson-2.13.1 & 2 & \\
527515025/JavaTest & geoip2-2.12.0 & geoip2-4.3.1 & 2 & \\
72crm/72crm-java & fastjson-1.2.54 & fastjson-2.0.57 & 80 & \\
1036225283/xws & fastjson-1.2.31 & fastjson-2.0.57 & 8 & \\
0RaymondJiang0/tushare-java & fastjson-1.2.23 & fastjson-2.0.57 & 14 & \\
72crm/72crm-java & druid-1.0.29 & druid-1.2.25 & 4 & \\
47degrees/firebrand & commons-logging-1.1.1 & commons-logging-1.3.5 & 3 & \\
2shou/HBaseObserver & & & 1 & \\
52North/triturus & & & 3 & \\
0RaymondJiang0/tushare-java & commons-csv-1.3 & commons-csv-1.14.0 & 2 & \\
9527dong/tiny-spring & commons-beanutils-1.9.3 & commons-beanutils-1.11.0 & 1 & \\
47degrees/firebrand & commons-beanutils-1.8.0 & commons-beanutils-1.11.0 & 10 & \\
925781609/pattern & cglib-2.2.2 & cglib-3.3.0 & 4 & \\
9527dong/tiny-spring & & & 2 & 1 \\
0xdecaf/beam-enrichment-patterns & beam-sdks-java-io-jdbc-2.9.0 & beam-sdks-java-io-jdbc-2.65.0 & 3 & \\
0xdecaf/beam-enrichment-patterns & beam-sdks-java-io-google-cloud-platform-2.9.0 & beam-sdks-java-io-google-cloud-platform-2.65.0 & 3 & \\
0xdecaf/beam-enrichment-patterns & beam-sdks-java-core-2.9.0 & beam-sdks-java-core-2.65.0 & 4 & \\
\bottomrule
\end{tabular}
\end{table*}



\subsection{Writing Manual Test Cases}
We apply taint analysis in our tool to help filter out irrelevant answers. When we run our tool based only on a CHA-based call graph, which
searches for unchecked exceptions in the transitively called method bodies as well as in the entry method's body, we get
285 (as depicted in ~\ref{t}) callsites that potentially might have an unchecked exception that can cause a behavioural breaking change.

We initially tried to write test cases for those 285 cases but were often unable to write a test case that could trigger
the newly added unchecked exception. In most of the cases, we observed that the parameters responsible for triggering the 
exceptions were not the ones passed by the client to the library method.

As with the \texttt{protobuf} case in Section~\ref{sec:methodology}, which added a new-but-untriggerable unchecked exception, taint analysis played a crucial role in reducing the number of false positives.

\vspace{1em}
\begin{tcolorbox}[colback=gray!10, colframe=black]
By adding taint analysis, we reduced the number of potentially affected callsites from 285 to just 49.
\end{tcolorbox}
\vspace{1em}

To assess the real-world consequences of these remaining 49 callsites, we manually constructed test cases. In most cases, we were able to provide inputs that trigger the newly added exceptions, confirming that they represent real behavioural breaking changes.

In other cases, the exception was not triggered immediately because the client passed hardcoded values or had safeguards like null checks. While these do not cause immediate failures, they remain latent risks—future code changes could inadvertently expose the client to the newly added exceptions.

\vspace{1em}
\begin{tcolorbox}[colback=gray!10, colframe=black]
\textbf{Answer RQ2:} Yes, client applications do call methods with newly added unchecked exceptions. Out of 99 client-library pairs in our corpus, we identified 49 callsites that reached newly-added exceptions, distributed across 6 of our 36 clients. We were able to construct test cases that trigger the exception in some cases.
\end{tcolorbox}
\vspace{1em}

\subsection{Discussion: Developer-Facing Implications}

Behavioural breaking changes caused by unchecked exceptions during API evolution are particularly dangerous. Such changes do not show up at compile time, and they do not affect method signatures, which means that the existing tools that we are aware of cannot detect them. For instance, both japicmp and Revapi, widely used tools for detecting breaking changes, focus on syntactic differences in method signatures. While they can both flag checked exceptions—since they appear in method declarations—they do not analyze the method implementations, and thus have no way of identifying newly added unchecked exceptions. As a result, developers who rely solely on either japicmp or revapi could remain unaware of serious runtime-breaking issues.

Some tools have tried to tackle the challenge of behavioural breaking changes. CompCheck~\cite{CompCheck}, for example, works by identifying test cases in some clients and reusing them for others with similar API usage. But this approach depends heavily on the presence of thorough test suites. In practice, most clients do not have such comprehensive coverage, especially not for edge cases involving unchecked exceptions.

This is where UnCheckGuard steps in. Unlike existing work, it does not rely on existing test cases. Instead, it compares the old and new versions of a library using static analysis to detect newly added unchecked exceptions, and then runs taint analysis to filter out changes that do not affect the client. By avoiding the need for a test suite, it can reveal behavioural breaking changes that other tools overlook.

In doing so, UnCheckGuard addresses an important gap. It gives developers visibility into a class of breaking changes that are easy to miss but costly in practice—helping them catch potential failures early, before they reach production.
