% look for citations about library usage
The use of libraries developed by others is ubiquitous in modern
software development~\cite{huang22:_charac_java,wang20:_java}. Libraries enable developers to include
functionality in their own client software without having to
implement it themselves.  However, libraries developed by others are
also updated by others, on schedules that are not controlled by the client developers.

Especially when one is developing software that is exposed to the Internet, one
has a responsibility to incorporate security updates for the
libraries that one is using as a client~\cite{wu23:_under_threat_upstr_vulner_downs}, or else risk vulnerabilities
being exposed in one's software~\cite{haryono22:_autom_ident_librar_vulner_data,zhan21:_atvhun,alfadel23:_empir_python}. The obligation to update libraries is
a form of technical debt that accrues automatically with the passage
of time.

% should we cite our Onward 2020 paper here?
However, upgrading libraries is not painless~\cite{elizalde18:_towar_smoot_librar_migrat,derr17:_keep,dann23:_upcy}: new
versions of libraries may include breaking Application Programming
Interface (API) changes~\cite{dietrich14:_broken}, requiring developers to verify that their own client
code continues working with the new library versions. This is
inconvenient at best and can require nontrivial amounts software development at worst,
often without the reward of useful new features for the client software---reacting to upgrades
just allows the client software to continue working, in a hopefully less-vulnerable
state.

Compilers and simple static
checkers~\cite{brito18:_apidif,foo18:_effic_static_check_librar_updat}
can verify the absence of syntactic breaking changes in libraries,
e.g. changes to signatures of public methods, retractions of
formerly-existing methods, or even syntactic changes to library method
implementations. The situation is worse for semantic breaking changes:
there do not exist techniques for reliably detecting such changes. Of
course, in its full generality, the problem is undecidable, though
breaking change detection can be estimated using static and dynamic program analysis
techniques.

In this work, we contribute a way to detect one type of potentially breaking
change in a library. Our work enables client developers to identify changes
to the set of exceptions that may be thrown by a Java library, particularly
by the APIs that are actually used by client code. A new exception thrown by a library
constitutes a breaking change; uncaught exceptions can cause the client to crash or
to exhibit unexpected behaviour.

Although developers tend to ignore even checked
exceptions~\cite{nakshatri16:_analy_java}, we contend that incrementally informing
developers only about newly-added exceptions is likely to be more tractable, consistent with the
design principles of Google's Tricorder tool~\cite{sadowski15:_tricor}.
Thus, we leverage taint analysis
to reduce the number of irrelevant reports that we report to client developers.
We aim to show only changed library APIs that may realistically throw new exceptions
in updated versions, minimizing the number of false positives~\cite{pashchenko20:_vuln4,pashchenko18:_vulner}.
We hope that our reports enable developers to better understand how new exceptions affect their client code,
possibly by writing test cases for clients that trigger these exceptions 

We explore the following research questions:

\noindent
{\bf RQ1.} How often do published changes to Java libraries throw new unchecked exceptions in methods,
and under what circumstances do such exceptions occur (e.g. major/minor/patch versions)?

\noindent
{\bf RQ2.} Do library clients, in practice, call methods with new added exceptions, and is it possible for the clients to trigger these exceptions? Is it possible to write client test cases that trigger the exceptions?

In our corpus of 68 distinct libraries, we found 19 libraries with newly-added exceptions, including exceptions that are added in minor and patch releases.
We then investigated 98 client-library pairs to explore the prevalence of potentially breaking behavioural changes.
We found that new potentially client-relevant unchecked exceptions occured in 8 of the 68 libraries, and that clients called methods containing these exceptions at 14 client callsites.
This shows that client applications do in fact call library methods that throw these new exceptions.
Furthermore, we demonstrated that it is possible to trigger some of these exceptions by writing test cases using client code.

The contributions of this work are as follows:

\begin{itemize}[noitemsep]
\item We conduct an empirical study of libraries and real-world client-library pairs to assess the prevalence of behavioural breaking changes caused by newly added unchecked exceptions.
\item We implement the \textit{UnCheckGuard} static analysis tool, which combines RTA-based call graph construction with taint analysis to detect and verify the relevance of newly added unchecked exceptions.
\item We evaluate 98 client-library pairs from the DUETS dataset~\cite{durieux21:_duets} using \textit{UnCheckGuard}, identifying 14 call sites where newly added unchecked exceptions could cause behavioural breaking changes, and write test cases showing that the exceptions can be triggered from client code.
\end{itemize}


%RQ1: Do libraries have breaking changes in the form of added exceptions?
%RQ2: Do clients call the methods which have these breaking changes, and can we trigger the new exceptions?

%true, [RQ1] I think even logically it would make sense to showcase that libraries do add new unchecked exceptions (UE). We can even do something about the version in which they are adding the UE (do they introduce the UE in major version or minor version or patch version). [RQ2] Then we can follow it up with how often do clients call such method which have a newly added UE and can we trigger those UE (the taint analysis part can be defended along with it). 

% https://hasel.auckland.ac.nz/2023/11/12/understanding-breaking-changes-in-the-wild/
