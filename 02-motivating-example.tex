We continue with a motivating example drawn from the DUETS collection~\cite{durieux21:_duets}
of client/library pairs. 
Our example exhibits a case
where a revision of the \texttt{httpcore} library adds a check for an
error condition.  If the condition holds, the library method will
explicitly throw an \texttt{IllegalArgumentException}. The client, \texttt{HttpAsyncClientUtils},
calls the relevant part of the library, and thus may be affected by the new exception.

\paragraph{Library} Specifically, all constructors for the \texttt{org.apache.http.HttpHost} class transitively call
the static method \texttt{Args.containsNoBlanks()}. Between version 4.4.6 and version 4.4.16, the \texttt{httpcore}
developers added the following lines of code to \texttt{containsNoBlanks()}:
\begin{lstlisting}[language=Java]
  if (argument.length() == 0) {
    throw new IllegalArgumentException
      (name + `` may not be empty'');
  }
\end{lstlisting}
All \texttt{HttpHost} constructors take a \texttt{hostname} parameter and call \texttt{containsNoBlanks()}
with that parameter (to check that it contains no blanks). It is therefore possible to trigger this newly-thrown
exception in a client by attempting to instantiate a new \texttt{HttpHost} object and passing it an empty
\texttt{hostname}.

Our UnCheckGuard tool analyzes the change in \texttt{httpcore} and reports that, in
version 4.4.16, all of the \texttt{HttpHost} constructors may now throw an
\texttt{IllegalArgumentException} via the \texttt{containsNoBlanks()} method, which
was not thrown in 4.4.6.

To detect this change, UnCheckGuard accepts JAR files for both \texttt{httpcore-4.4.6} and \texttt{httpcore-4.4.16}, along with the entry method used by the client---here, \texttt{<Util.HttpClientUtil.HttpAsyncClient: org.apache.http.impl.nio.client. CloseableHttpAsyncClient createAsyncClient(boolean)>}. It uses SootUp~\cite{Karakaya24:_sootup} to construct an RTA-based call graph~\cite{bacon96:_fast_static_analy_c_virtual_funct_calls} starting from this method and identifies all transitively reachable calls. UnCheckGuard then collects all unchecked exceptions thrown within this graph and applies taint analysis using FlowDroid~\cite{Arzt14:_flowdroid} to check whether method parameters controlled by the client can reach the exception sites. In version 4.4.6, it finds two sites throwing \texttt{IllegalArgumentException}, while in 4.4.16, it detects three—each of which the client can potentially trigger using the values it passes as parameters\todo{TBR}.

\paragraph{Client} A newly-added exception is only relevant to a client if the client may potentially
trigger that exception. It turns out that our \texttt{HttpAsyncClientUtils} client has reachable code
that creates an \texttt{HttpHost} with an empty \texttt{host}. The
public \texttt{HttpAsyncClient.createAsyncClient()} method
takes a \texttt{proxy} parameter and contains the following code:
\begin{lstlisting}[language=Java,basicstyle=\scriptsize\ttfamily]
 if (proxy) {
  return HttpAsyncClients.custom()
   .setConnectionManager(conMgr)
   .setDefaultCredentialsProvider(credentialsProvider)
   .setDefaultAuthSchemeRegistry(authSchemeRegistry)
   .setProxy(new HttpHost(host, port))
   .setDefaultCookieStore(new BasicCookieStore())
   .setDefaultRequestConfig(requestConfig).build();
 } else {
   // ...
\end{lstlisting}
where \texttt{host} is a private field initialized to the empty string.
Thus, calling \texttt{createAsyncClient(true)} triggers an exception when executed with
\texttt{httpcore} version 4.4.16 but not with 4.4.6.

To detect that our \texttt{HttpAsyncClientUtils} client calls a method from \texttt{httpcore-4.4.6} which, upon upgrading to \texttt{httpcore-4.4.16}, may throw a new unchecked exception, UnCheckGuard begins by identifying all library methods invoked by the client. It then analyzes both the existing and the upgraded versions of the library. Using this analysis, it determines whether any newly introduced unchecked exceptions are reachable from the client's code.\todo{say more about what reachability means}

In this case, based on the confirmed reachability of the new exception, we report that the library-client pair \texttt{HttpAsyncClientUtils} and \texttt{httpcore} exhibits a behaviuoral breaking change.

Given this report, it is not difficult to write a test case that calls the client's \texttt{createAsyncClient()} method
and triggers the exception after an upgrade:
\begin{lstlisting}[language=Java,basicstyle=\scriptsize\ttfamily]
@Test
void testCreateAsyncClientThrowsExceptionForEmptyProxyHost() {
  HttpAsyncClient client = new HttpAsyncClient();

  IllegalArgumentException exception =
    assertThrows(IllegalArgumentException.class, () -> {
        client.createAsyncClient(true);
    });

    assertTrue(exception.getMessage()
    .contains("may not be empty"),
      "Expected exception due to empty hostname "+
      "after upgrading to httpcore-4.4.16");
}
\end{lstlisting}




