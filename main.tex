% !TeX root = main.tex
\documentclass[conference]{IEEEtran}
\IEEEoverridecommandlockouts
% The preceding line is only needed to identify funding in the first footnote. If that is unneeded, please comment it out.
\usepackage[utf8]{inputenc} % allow utf-8 input
\usepackage[T1]{fontenc}    % use 8-bit T1 fonts
\usepackage{hyperref}       % hyperlinks
\usepackage{url}            % simple URL typesetting
\usepackage{booktabs}       % professional-quality tables
\usepackage{nicefrac}       % compact symbols for 1/2, etc.
\usepackage{microtype}      % microtypography
\usepackage{lipsum}
\usepackage{fancyhdr}       % header
\usepackage{graphicx}       % graphics
\usepackage{bookmark}
\usepackage{enumitem}

%\usepackage{caption}
\usepackage{svg}
\usepackage{cite}
\usepackage{amsmath,amssymb,amsfonts}
\usepackage{algorithmic}
\usepackage{textcomp}
\usepackage{xcolor}
\usepackage{flushend}
\usepackage{mathtools}
\usepackage[square,sort,comma,numbers]{natbib}
\bibliographystyle{IEEEtranN}
\usepackage{tabularray}
\usepackage{tikz}
\usetikzlibrary{arrows.meta, positioning}
\usepackage{subfig}
\usepackage{listings}
% \usepackage{todonotes}
\usepackage{tabularx}
\usepackage{multirow}
\usepackage{pdfpages}
\usepackage{placeins}
\usepackage{cellspace}
\usepackage{todonotes}
\usepackage{mdframed}
\usepackage{makecell}
\usepackage[most]{tcolorbox}

\newtcolorbox{findingbox}[1][]{
  colback=gray!10!white,
  colframe=gray!50!black,
  #1
}


\setlength\cellspacetoplimit{3.5pt}
\setlength\cellspacebottomlimit{3.5pt}

% \def\BibTeX{{\rm B\kern-.05em{\sc i\kern-.025em b}\kern-.08em
%     T\kern-.1667em\lower.7ex\hbox{E}\kern-.125emX}}
\begin{document}

	\title {Detecting Exception-Related Behavioural Breaking Changes with UnCheckGuard}

    %% \author{\IEEEauthorblockN{Vinayak Sharma\IEEEauthorrefmark{1}, Patrick Lam\IEEEauthorrefmark{2}}
    %% \IEEEauthorblockA{\IEEEauthorrefmark{1}University of Waterloo; \emph{vinayak.sharma1@uwaterloo.ca}}
    %% \IEEEauthorblockA{\IEEEauthorrefmark{2}University of Waterloo; \emph{patrick.lam@uwaterloo.ca}}
    %% }

    \maketitle
    \thispagestyle{plain}
    \pagestyle{plain}

    \begin{abstract}
      The ubiquitous use of third-party libraries in software development has enabled developers to quickly add
      new functionality to their client software. Unfortunately, library usage also carries a cost in
      terms of software maintenance: library upgrades may include breaking changes, in which expectations
      from the client about library behaviour are no longer met in new library versions. Behavioural breaking
      changes can be particularly insidious, and in their full generality, could require sophisticated program
      analysis techniques to (approximately) detect.

      In this work, we present our UnCheckGuard tool, which detects a class of behavioural breaking changes---those
      related to exceptions thrown by Java libraries. UnCheckGuard analyzes both sides of the library/client
      duet. On the library side, UnCheckGuard creates a list of new exceptions that may be thrown by methods
      in a library's public API, including by its transitive callees. On the client side, UnCheckGuard identifies
      client methods that call library methods with new exceptions. To reduce false positives, UnCheckGuard
      additionally filters out client methods that cannot trigger these new exceptions. It therefore can be
      used by client developers as a tool to screen library updates for relevant incompatibilities.

      We have evaluated UnCheckGuard on 68 libraries and 98 library-client pairs drawn from the DUETS collection
      and found 8 libraries with newly-added exceptions, as well as 14 callsites to library methods which,
      when upgraded to the latest version, may introduce
      a behavioural breaking change in the client due to a newly added unchecked exception. These findings
      highlight the practical value of UnCheckGuard in identifying exception-related incompatibilities
      introduced by library upgrades.


    \end{abstract}
    
    

    \begin{IEEEkeywords}
        client/library interactions, behavioural breaking changes, exceptions, static analysis
    \end{IEEEkeywords}


    \section{Introduction} \label{sec:introduction}
    % look for citations about library usage
The use of libraries developed by others is ubiquitous in modern
software development. Libraries enable developers to include
functionality in their own client software without having to
implement it themselves.  However, libraries developed by others are
also updated by others, on schedules that are not controlled by the client developers.

When one is developing software that is exposed to the Internet, one
has a responsibility to incorporate at least security updates for the
libraries that one is using as a client, or else risk vulnerabilities
being exposed in one's software. The obligation to update libraries is
a form of technical debt that accrues automatically with the passage
of time.

% should we cite our Onward 2020 paper here?
However, client developers are reluctant to upgrade libraries: new
versions of libraries may include breaking Application Programming
Interface (API) changes, requiring developers to verify that their own
code continues working with the new library versions. This is
inconvenient at best.

Compilers and simple static checkers can verify the absence of
syntactic breaking changes in libraries, e.g. changes to signatures of
public methods, or retractions of formerly-existing methods. The
situation is worse for semantic breaking changes: there do not exist
techniques for reliably detecting such changes. Of course, in its full
generality, the problem is undecidable, though breaking change
detection can be estimated using program analysis techniques.

% cite FlowDroid

In this work, we contribute a way to detect one type of potentially breaking
change in a library. Our work enables client developers to identify changes
to the set of exceptions that may be thrown by a Java library, particularly
by the APIs that are actually used by client code. We leverage taint analysis
to reduce the number of false positives that we report to client developers,
and show only changed APIs that may realistically throw new exceptions
in updated versions. % something about making it easy to show test cases.

We explore the following research questions:

\noindent
{\bf RQ1.} How often do published changes to Java libraries include new added exceptions,
and under what circumstances do such exceptions occur (e.g. major/minor/patch versions)?

\noindent
{\bf RQ2.} Do library clients, in practice, call methods with new added exceptions, and is it possible for the clients to trigger these exceptions? Is it possible to write client test cases that trigger the exceptions?

\vspace*{1em}

Corpus. Summary of methodology.

The contributions of this work are as follows: \todo{clean this up}

\begin{itemize}[noitemsep]
\item empirical study of library usage
\item technique for detecting and implementation thereof
  \item results from technique
\end{itemize}

%RQ1: Do libraries have breaking changes in the form of added exceptions?
%RQ2: Do clients call the methods which have these breaking changes, and can we trigger the new exceptions?

%true, [RQ1] I think even logically it would make sense to showcase that libraries do add new unchecked exceptions (UE). We can even do something about the version in which they are adding the UE (do they introduce the UE in major version or minor version or patch version). [RQ2] Then we can follow it up with how often do clients call such method which have a newly added UE and can we trigger those UE (the taint analysis part can be defended along with it). 

% https://hasel.auckland.ac.nz/2023/11/12/understanding-breaking-changes-in-the-wild/


    \section{Motivating Example} \label{sec:motivating}
    We continue with a motivating example drawn from the DUETS collection~\cite{durieux21:_duets}
of client/library pairs. 
In our example,
a revision of the \texttt{httpcore} library adds a check for an
error condition.  If the condition holds, the library method will
explicitly throw an \texttt{IllegalArgumentException}. The client, \texttt{HttpAsyncClientUtils}\footnote{\url{github.com/a63881763/HttpAsyncClientUtils}},
calls the relevant part of the library, and thus may be affected by the new exception.

In the DUETS suite, our client, \texttt{HttpAsyncClientUtils}, declares a dependency on
version 4.4.6 of the \texttt{httpcore} library. Between the release of DUETS and today, the \texttt{httpcore} developers
have released a number of new versions, and at the time of writing, the latest version of \texttt{httpcore}
is 4.4.16.

\paragraph{Library} All constructors for the \texttt{org.apache.http.HttpHost} class transitively call
the static method \texttt{Args.containsNoBlanks()}. Between version 4.4.6 and version 4.4.16, the \texttt{httpcore}
developers added the following lines of code to \texttt{containsNoBlanks()}:
\begin{lstlisting}[language=Java]
  if (argument.length() == 0) {
    throw new IllegalArgumentException
      (name + `` may not be empty'');
  }
\end{lstlisting}
Specifically, all \texttt{HttpHost} constructors take a \texttt{hostname} parameter and call \texttt{containsNoBlanks()}
with that parameter (to check that it contains no blanks). It is therefore possible to trigger this newly-thrown
exception in a client by attempting to instantiate a new \texttt{HttpHost} object and passing it an empty
\texttt{hostname}.

Our UnCheckGuard tool analyzes the change in \texttt{httpcore} and reports that, in
version 4.4.16, all of the \texttt{HttpHost} constructors may now throw an
\texttt{IllegalArgumentException} via the \texttt{containsNoBlanks()} method.
This exception was not thrown in 4.4.6.

To detect this change, UnCheckGuard processes JAR files for both \texttt{httpcore-4.4.6} and \texttt{httpcore-4.4.16}. It uses SootUp~\cite{Karakaya24:_sootup} to construct an RTA-based call graph~\cite{bacon96:_fast_static_analy_c_virtual_funct_calls} starting from the public \texttt{<init>(String, int)}\footnote{Specifically, method \texttt{<init>(String, int)} returning a \texttt{void} on class \texttt{org.apache.http.HttpHost}} method and identifies the set of all transitively reachable methods. It starts from the public \texttt{<init>(String, int)} method as that is the method used by the client from \texttt{htttpcore} library. UnCheckGuard then collects all unchecked exceptions thrown within this set and applies taint analysis using FlowDroid~\cite{Arzt14:_flowdroid} to check whether method parameters controlled by the client can reach the exception sites. If the exception is unreachable from client code (in terms of taint analysis), then the client cannot supply values to the library which will trigger the exception. In version 4.4.6, it finds two sites throwing \texttt{IllegalArgumentException}, while in 4.4.16, it detects three—each of which the client can potentially trigger using the values it chooses to pass to the library as parameters.

\paragraph{Client} 
A newly-added exception is only relevant to a client if the client may
potentially trigger that exception.  It turns out that
our \texttt{HttpAsyncClientUtils} client has reachable code from the public \texttt{createAsyncClient(boolean)}\footnote{Fully-qualified: method \texttt{createAsyncClient(boolean)} returning a \texttt{CloseableHttpAsyncClient} on class \texttt{Util.HttpClientUtil.HttpAsyncClient}.} method
that creates an \texttt{HttpHost} with an empty \texttt{host}. The
public \texttt{createAsyncClient(boolean)} method takes a \texttt{proxy}
parameter and contains the following code:
\begin{lstlisting}[language=Java,basicstyle=\scriptsize\ttfamily]
 if (proxy) {
  return HttpAsyncClients.custom()
   .setConnectionManager(conMgr)
   .setDefaultCredentialsProvider(credentialsProvider)
   .setDefaultAuthSchemeRegistry(authSchemeRegistry)
   .setProxy(new HttpHost(host, port))
   .setDefaultCookieStore(new BasicCookieStore())
   .setDefaultRequestConfig(requestConfig).build();
 } else {
   // ...
\end{lstlisting}
where \texttt{host} is a private field initialized to the empty string.
Thus, calling \texttt{createAsyncClient(true)} triggers an exception when executed with
\texttt{httpcore} version 4.4.16 but not with 4.4.6.

To detect that our \texttt{HttpAsyncClientUtils} client calls a method from \texttt{httpcore-4.4.6} which, upon upgrading to \texttt{httpcore-4.4.16}, may throw a new unchecked exception, UnCheckGuard begins by identifying all external library methods invoked anywhere in the client. It then analyzes both the existing and the upgraded versions of the library. Using this analysis, it determines whether any newly introduced unchecked exceptions are reachable from the client's code. Here, reachability means that the client can trigger the exception in the library on some execution of the program, using values it passes to the library as parameters.

In order to check if the client-supplied values can reach the exception-throwing site, we use taint analysis. Taint analysis is essential in this scenario because a path from the client callsite to an exception-throwing statement is not sufficient to conclude that the exception is actually triggerable by the client. Many such paths may exist, but the control-flow conditions leading to the exception might depend entirely on internal library values, rather than on client-supplied inputs. Therefore, taint analysis becomes necessary to distinguish actual behavioural breaking changes from false positives.

Taint analysis helps by tracking whether any client-supplied value (source) can propagate to the conditional or operation that triggers the exception (sink). If taint analysis determines that no client-supplied input flows into the exception-triggering logic, then we can conclude that the newly added exception will not cause a behavioural breaking change.

This implies that a path exists in the interprocedural control-flow graph (which one can compute with Class Hierarchy Analysis (CHA)) from the client method to the statement throwing the exception in the library.

In this case, based on the confirmed reachability of the new exception, we report that the library-client pair \texttt{HttpAsyncClientUtils} and \texttt{httpcore} exhibits a behavioural breaking change.

Given this report, it is straightforward to write a test case that calls the client's \texttt{createAsyncClient()} method
and triggers the exception after an upgrade:
\begin{lstlisting}[language=Java,basicstyle=\scriptsize\ttfamily]
@Test
void testCreateAsyncClientThrowsExceptionForEmptyProxyHost() {
  HttpAsyncClient client = new HttpAsyncClient();

  IllegalArgumentException exception =
    assertThrows(IllegalArgumentException.class, () -> {
        client.createAsyncClient(true);
    });

    assertTrue(exception.getMessage()
    .contains("may not be empty"),
      "Expected exception due to empty hostname "+
      "after upgrading to httpcore-4.4.16");
}
\end{lstlisting}

    %\section{Background}\label{sec:background}
    %\input{background}


    \section{Data Collection}\label{sec:data-collection}
    This section describes the systematic approach we used to construct the dataset for our study on behavioral incompatibilities caused by newly added unchecked exceptions in upgraded Java libraries. 

Broadly, we require three key components: Java-based clients that depend on third-party libraries, the current versions of the libraries used by these clients, and the latest available versions of those same libraries. For each client-library pair, we also need to extract the set of unchecked exceptions thrown by the library methods actually used by the client.

To obtain this information, our methodology involves several key steps: identifying suitable Java clients, extracting their library dependencies, resolving both the current and latest versions of the libraries, analyzing exception behavior in both versions, and recording all methods that introduce newly added unchecked exceptions. 

Each step is essential for enabling our tool, UnCheckGuard, to pinpoint client call sites that may be affected by behavioral breaking changes.

\subsection{Collecting Clients}

To begin our analysis, we first collected suitable client projects. We used the DUETS dataset~\cite{durieux21:_duets}, which provides a curated list of Java-based clients hosted on GitHub, each with at least five stars. We attempted to download each client repository and discarded any client that failed to download.

Next, we checked whether the project included a \texttt{pom.xml} file, which indicates that it is a Maven-based project. This step was essential, as our analysis depends on running Maven commands. We compiled each client to produce a JAR file and kept only those clients that compiled successfully for further analysis.

\subsection{Library Version Resolution}

A core component of our tool relies on analyzing both the version of the library currently used by the client and the latest available version. To collect the current version, we run the Maven command \texttt{mvn dependency:copy-dependencies}, which downloads all the dependencies declared in the client's build configuration.

To obtain the latest versions of these dependencies, we run the following Maven command:
\begin{lstlisting}[language=bash, breaklines=true, basicstyle=\ttfamily\small]
mvn org.codehaus.mojo:versions-maven-plugin:2.18.0:use-latest-versions
\end{lstlisting}
This command updates the \texttt{pom.xml} file with the most recent versions of all declared dependencies. We then re-run \texttt{mvn dependency:copy-dependencies} to download the updated set of libraries.

By following this process, we obtain both the original and the upgraded versions of each library used by the client, enabling us to perform a comparative analysis of behavioral changes across versions.




    \section{Methodology}\label{sec:methodology}
    We describe a methodology for detecting and verifying newly added unchecked exceptions in a library when it is updated from an older version to a newer one. Our focus is on identifying the impact of such changes on client code. Specifically, we analyze client programs to detect usage of library methods that now throw previously non-existent unchecked exceptions. Java distinguishes between checked exceptions, which appear as part of method signatures, and unchecked exceptions, which do not. Unchecked exceptions may therefore introduce a class of breaking changes that signature-based syntactic approaches for Java do not detect.

We begin by collecting both the client programs and the necessary information to perform our analysis. Specifically, we extract all external library methods invoked by the client. For each of these methods, we analyze their implementation in both the version of the library currently used by the client and the latest version. This allows us to compare their behaviour across versions. If we find that a method now throws a newly added unchecked exception in the latest version, and that the exception can be triggered from the client code, we report a potential behavioural breaking change. To verify whether this change is in fact breaking, we currently manually write test cases to verify that the client may be affected by the newly introduced exception.

\subsection{Analysis Setup}

The analysis setup is divided into two main phases: client selection and knowledge extraction for analysis.

\subsubsection{Client Selection}

We began by collecting clients to evaluate our approach. For this, we used the DUETS dataset~\cite{durieux21:_duets}, which provides a curated set of Java-based client programs available on GitHub, each with at least five stars. The first step involved compiling each client's source code to generate the corresponding JAR files required for further analysis. Clients that failed to compile were discarded, as the lack of a compiled artifact prevents further inspection.

All clients were compiled using Java 11. To ensure compatibility with our analysis tooling, we checked whether each project contained a \texttt{pom.xml} file, indicating that it is a Maven-based project. Since our workflow relies on Maven for dependency resolution and build automation, only clients that could be built using Maven were included in our dataset.

\subsubsection{Knowledge Extraction for Analysis}

Once the valid clients were selected, we used SootUp~\cite{Karakaya24:_sootup} to extract all external method calls made by each client. This was accomplished by traversing all methods in the client and identifying occurrences of \texttt{JInvokeStmt}, which represent method invocation statements. In parallel, we collected both the old and new versions of the libraries used by the client, obtaining the corresponding JAR files for each version. These libraries were also analyzed using SootUp to extract all available method signatures.

By correlating the external method calls from the client with the methods defined in each library, we constructed an accurate mapping between client calls and the corresponding library methods. This mapping enabled us to track how the behaviour of these methods evolved across versions.

As a result of this process, we obtained the JAR files for both the old and new versions of each library, along with a filtered list of external methods that are candidates for further analysis. These methods form the basis for detecting newly introduced unchecked exceptions that may lead to behavioural breaking changes in the client.

\subsection{Finding Newly Added Unchecked Exceptions}

Our goal is to detect whether upgrading a library introduces new unchecked exceptions that could affect client behaviour. To achieve this, we divide the process into two stages: first, identifying newly added unchecked exceptions using call graph analysis; and second, verifying their reachability from client input using taint analysis.

\subsubsection{Exception Discovery with RTA}

To detect newly added unchecked exceptions in the upgraded library, we first construct a call graph using \textbf{Rapid Type Analysis (RTA)} via SootUp~\cite{Karakaya24:_sootup}. We use RTA instead of \textbf{Class Hierarchy Analysis (CHA)} because RTA provides a more precise approximation of runtime behaviour. While CHA includes all methods defined in subclasses and interface implementations regardless of whether they are actually invoked, RTA considers only those types that are instantiated in the program, resulting in a more accurate and context-sensitive call graph.

In practice, CHA tends to over-approximate and report unreachable method calls. For example, in one case, CHA identified a path from the method \texttt{<com.alibaba.fastjson.JSONObject: java.lang.String getString (java.lang.String)>}, reporting an exception thrown in the \texttt{JSONObject} constructor as reachable. However, manual inspection revealed that this path was spurious—the method \texttt{getString} never actually reached the constructor in question. Using RTA eliminated such false positives by restricting analysis to only concretely instantiated types.

After building the RTA-based call graph, we traverse the reachable paths for each client-used method and collect all exceptions that are subclasses of \texttt{java.lang.RuntimeException} or \texttt{java.lang.Error}. These represent the complete set of unchecked exceptions that the client might be newly exposed to due to the library upgrade.

\subsubsection{Exception Verification with Taint Analysis}

Once the list of unchecked exceptions has been collected, we proceed to verify whether they are truly reachable from client-supplied inputs. This verification step is performed using \textbf{FlowDroid}~\cite{Arzt14:_flowdroid}, a static taint analysis framework. 

To enable taint analysis, we generate a \textit{driver stub} for each client-used method. The purpose of this stub is to declare each method parameter as a taint source. For each parameter, we define a wrapper function whose return value is marked as a source. The driver stub generation process is automated using SootUp and handles a wide variety of cases, including:
\begin{itemize}
  \item Constructor methods (\texttt{<init>} using \texttt{new ClassName(...)})
  \item Static and instance methods
  \item Void and non-void return types
  \item Primitive parameters (e.g., \texttt{int} $\rightarrow$ \texttt{0})
  \item Object parameters (defaulted to \texttt{null})
  \item Nested classes (converting \texttt{\$} to \texttt{.})
  \item Multiple parameters (sources named \texttt{SourceN()}, where $N$ is the parameter index)
  \item Overloaded methods (only one version retained)
\end{itemize}

Each exception collected during the RTA phase is treated as a \textit{taint sink}. For this analysis, we intentionally use \textbf{Class Hierarchy Analysis (CHA)} to construct the call graph for FlowDroid. Although CHA is less precise than RTA, its conservative nature is beneficial in this context. It ensures broader coverage, helping us avoid false negatives where actual flows might be missed due to the strictness of RTA.

Consider the following example from the \texttt{beam-sdks-java-core} library. The method \texttt{GroupByKey.applicableTo(PCollection)} throws an \texttt{IllegalStateException} under a specific condition:
\begin{lstlisting}[language=Java,breaklines=true,breakatwhitespace=false,basicstyle=\scriptsize\ttfamily]
public static void applicableTo(PCollection<?> input) {
    WindowingStrategy<?, ?> windowingStrategy = input.getWindowingStrategy();
    if (windowingStrategy.getWindowFn() instanceof GlobalWindows
        && windowingStrategy.getTrigger() instanceof DefaultTrigger
        && input.isBounded() != IsBounded.BOUNDED) {
      throw new IllegalStateException(
          "GroupByKey cannot be applied to non-bounded PCollection in the GlobalWindow without a"
              + " trigger. Use a Window.into or Window.triggering transform prior to GroupByKey.");
    }
}
\end{lstlisting}

Here, the source is the method parameter \texttt{input}, and the sink is the \texttt{IllegalStateException}. Since the exception is conditionally thrown, RTA does not resolve the path from the input parameter to the exception. However, due to its conservative over-approximation, CHA includes the conditional path, allowing the taint analysis to correctly identify the exception as being influenced by client input.

This trade-off justifies the use of CHA in the verification phase: although it may over-approximate in some cases, it ensures that no critical exception flows are missed.


\subsection{Verification of Unchecked Exceptions}



    \section{Results}\label{sec:results}
    \todo{detailed results table}

We evaluated UnCheckGuard on Java-based clients from the DUETS dataset~\cite{durieux21:_duets}, selecting only clients with at least 5 stars on GitHub to ensure practical relevance.

The goal of our tool is to detect whether a client calls a library method that, upon upgrading the library to a newer version, introduces a previously non-existent unchecked exception—potentially resulting in a behavioural breaking change.

We explore the following research questions:

\begin{itemize}
  \item[\textbf{RQ1:}] How often do published changes to Java libraries include new added exceptions, and under what circumstances do such exceptions occur (e.g., major/minor/patch versions)?
  \item[\textbf{RQ2:}] Do library clients, in practice, call methods with new added exceptions, and is it possible for the clients to trigger these exceptions? Is it possible to write client test cases that trigger the exceptions?
\end{itemize}

\subsection{Addition of Exceptions in Java Libraries}

Our evaluation included 36 client applications, which depended on 83 distinct libraries. Across these, we formed 98 client-library pairs, each corresponding to a combination of a specific client and one of its used libraries. 

UnCheckGuard detected 142 callsites across these 98 pairs where the upgraded version of the library introduced a previously unseen unchecked exception. However, it was not possible to trigger all of these exceptions using the client's methods, even with a free choice of parameters to pass to the client code. We therefore applied a taint-based reachability analysis to identify only those cases that could result in actual runtime failures. After this filtering step, we identified 14 callsites in total—spanning 8 distinct libraries—that were definitively affected by a newly added unchecked exception. 

Notably, 6 out of these 8 libraries introduced such changes as part of a major version bump, which often signals breaking changes. However, we also observed one case each in a minor and a patch version upgrade. This indicates that even smaller upgrades may introduce behavioural breaking changes via unchecked exceptions—something developers may not anticipate.

\vspace{1em}
\begin{tcolorbox}[colback=gray!10, colframe=black]
\textbf{Answer RQ1:} Java libraries introduce newly added unchecked client-relevant exceptions across versions frequently enough to be relevant to clients. Out of 98 client-library pairs, we identified 14 callsites affected by newly added unchecked exceptions across 8 libraries (8\%). These changes occurred in major version upgrades (6 times) but also in minor (1) and even patch (1) version upgrades (e.g., \texttt{httpcore-4.4.6}~$\rightarrow$~\texttt{httpcore-4.4.16}).
\end{tcolorbox}
\vspace{1em}

\begin{table}[h]
\centering
\caption{Distribution of Breaking Changes Across Version Types}
\label{tab:version-distribution}
\begin{tabular}{lcc}
\toprule
\textbf{Version Type} & \textbf{Libraries} & \textbf{Affected Call Sites} \\
\midrule
Major Version Change & 6 & 11 \\
Minor Version Change & 1 & 2 \\
Patch Version Change & 1 & 1 \\
\bottomrule
\end{tabular}
\end{table}

\subsection{Effectiveness of Taint Analysis}

We apply Taint Analysis as a verification method in our tool. When we run our tool based only on an RTA-based call graph, which
searches for unchecked exceptions in the transitively called method bodies as well as in the entry method's body, we get
142 callsites that potentially might have an unchecked exception that can cause a behavioural breaking change.

We initially tried to write test cases for those 142 cases but were unable to write a test case that could trigger
the newly added unchecked exception. In most of the cases, we observed that the parameters responsible for triggering the 
exceptions were not the ones passed by the client to the library method.

We have discussed a case in Section~\ref{sec:methodology} where a newly added unchecked exception was present, but the client-supplied input was unable to trigger it. In that case, as well as in other similar instances we observed, taint analysis played a crucial role in eliminating such false positives.

\vspace{1em}
\begin{tcolorbox}[colback=gray!10, colframe=black]
By adding taint analysis, we reduced the number of potentially affected callsites from 142 to just 14.
\end{tcolorbox}
\vspace{1em}


\subsection{Impact on Clients}

UnCheckGuard not only identifies newly added exceptions but also helps assess their practical consequences for clients. In total, across all client-library pairs, we observed 8047 external method invocations. Among these, 142 targeted methods that throw a new unchecked exception in the upgraded version. After applying reachability filtering, 14 of these callsites were confirmed to potentially have a behavioural breaking change. 

To further evaluate the real-world impact of these changes, we attempted to construct test cases that could trigger the exception at runtime. In most of these cases, we successfully created such test inputs, confirming that the exception could propagate to the client. This validates that these changes are not merely theoretical but represent concrete runtime risks.

In a few instances, our analysis identified callsites where the client code either passed a hardcoded value or included precondition checks (e.g., null checks) that prevented the exception from being triggered under current conditions. While these cases did not result in immediate runtime failures, they remain important for developers to monitor. Future code changes—for instance, removing a check or altering the passed argument—could unknowingly expose these callsites to the newly added exception, leading to a breaking change.
\todo{we should make data available including the cases - pending}

\vspace{1em}
\begin{tcolorbox}[colback=gray!10, colframe=black]
\textbf{Answer RQ2:} Yes, client applications do call methods with newly added unchecked exceptions. We confirmed 14 such callsites across our corpus. We were able to construct test cases that trigger the exception. In others, the exception was guarded by a hardcoded value or a conditional check, but these callsites remain latent risks that could become active under future client code modifications.
\end{tcolorbox}
\vspace{1em}

\subsection{Developer-Facing Implications}

Behavioural breaking changes caused by unchecked exceptions during API evolution are particularly dangerous. Such changes do not show up at compile time, and they do not affect method signatures, which means that the existing tools that we are aware of cannot detect them. For instance, japicmp\footnote{https://github.com/siom79/japicmp}, a widely used tool for detecting breaking changes, focuses only on syntactic differences in method signatures. While it can flag checked exceptions—since they appear in method declarations—it has no way of identifying newly added unchecked exceptions. As a result, developers who rely solely on japicmp could remain unaware of serious runtime-breaking issues.

Some tools have tried to tackle the challenge of behavioural breaking changes. CompCheck~\cite{CompCheck}, for example, works by identifying test cases in some clients and reusing them for others with similar API usage. But this approach depends heavily on the presence of thorough test suites. In practice, most clients do not have such comprehensive coverage, especially not for edge cases involving unchecked exceptions.

This is where UnCheckGuard steps in. Unlike existing work, it does not rely on existing test cases. Instead, it compares the old and new versions of a library using static analysis to detect newly added unchecked exceptions, and then runs taint analysis to verify which ones might actually affect the client. By avoiding the need for a test suite, it can reveal behavioural breaking changes that other tools overlook.

In doing so, UnCheckGuard addresses an important gap. It gives developers visibility into a class of breaking changes that are easy to miss but costly in practice—helping them catch potential failures early, before they reach production.



    % TODO: add discussion
    % TODO: add threats to validity

    %\section{Threats to Validity}\label{sec:threats}
    %\input{threats}
    
    \section{Related Work}\label{sec:related-work}
    While much program analysis research considers a single version of a software artifact, some related work treats changes between versions, and we discuss some related work in that area. We also discuss empirical efforts to detect and empirically survey the prevalence of breaking changes.

Logozzo et al~\cite{logozzo14:_verif_modul_version} proposed the concept of verification modulo versions. Like us, verification modulo versions recognizes that program verification need to recognize that software evolves over time and that verification tools must take this into account---in particular, a developer often wants to know about potential verification issues unique to new code, rather than re-triaging issues previously reported. A fundamental difference between their work and ours is that we put the interface between the client and the library at the centre of our approach, and ensure that changes in the library must be visible to the client before we report them, while the verification modulo versions approach aims to detect behavioural differences between two versions of some software.

Møller et al~\cite{møller20:_detec_locat_javas_progr_affec} propose a domain-specific language for JavaScript library developers to use to indicate to client developers what has changed in a new version of their library. Our work addresses a specific subset of the breaking changes problem but automatically deduces changes in the library that are relevant to a particular client. It does not require additional work on the part of the library developer. More generally, and at the same time, Lam et al~\cite{lam20:_puttin_seman_seman_version} proposed the development of semantic version calculators, including the usage of both traditional and lightweight contracts for libraries, to allow library developers to declare, and client developers to understand, the impact of potential breaking changes in libraries.

Jayasuriya et al~\cite{jayasuriya23:_under_break_chang_wild,jayasuriya24} investigate the prevalence of breaking changes in the wild. In principle, under semantic versioning~\cite{preston-werner23:_seman_version}, library developers ought to indicate breaking changes by incrementing the major version number (i.e. the first number in the version triplet); however, Jayasuriya et al found that 41.58\% of (syntactic) breaking changes were not identified as such, and that 11.58\% of changes were beaking.

We have proposed a static approach to detecting breaking changes. Mujahid et al~\cite{mujahid20:_using_other_tests_ident_break_updat} proposed a dynamic approach to this problem. Their goal is to answer the question of whether a new version includes breaking changes or not, and they combine tests from ``the crowd'' (a collection of other projects) to decide the question, finding that such tests indicated a breaking change 60\% of the time. Our approach is much more specific to a specific library/client pair, and aims to detect if library $X$'s upgrade may break client $Y$. More like us, Jayasuriya et al~\cite{jayasuriya24:_under_apis} also use a dynamic approach (compared to our static approach) on a client/library pair to detect behavioral breaking changes in the client using its tests, finding that 2.30\% of library updates broke the client.

    

    \section{Conclusion}\label{sec:conclusion-&-future-work}
    In this work, we demonstrated the working of our tool UnCheckGuard
    
    \bibliography{references}

\end{document}
