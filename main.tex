% !TeX root = main.tex
\documentclass[conference]{IEEEtran}
\IEEEoverridecommandlockouts
% The preceding line is only needed to identify funding in the first footnote. If that is unneeded, please comment it out.
\usepackage[utf8]{inputenc} % allow utf-8 input
\usepackage[T1]{fontenc}    % use 8-bit T1 fonts
\usepackage{hyperref}       % hyperlinks
\usepackage{url}            % simple URL typesetting
\usepackage{booktabs}       % professional-quality tables
\usepackage{nicefrac}       % compact symbols for 1/2, etc.
\usepackage{microtype}      % microtypography
\usepackage{lipsum}
\usepackage{fancyhdr}       % header
\usepackage{graphicx}       % graphics
\usepackage{bookmark}
\usepackage{enumitem}

%\usepackage{caption}
\usepackage{svg}
\usepackage{cite}
\usepackage{amsmath,amssymb,amsfonts}
\usepackage{algorithmic}
\usepackage{textcomp}
\usepackage{xcolor}
\usepackage{flushend}
\usepackage{mathtools}
\usepackage[square,sort,comma,numbers]{natbib}
\bibliographystyle{IEEEtranN}
\usepackage{tabularray}
\usepackage{tikz}
\usetikzlibrary{arrows.meta, positioning}
\usepackage{subfig}
\usepackage{listings}
% \usepackage{todonotes}
\usepackage{tabularx}
\usepackage{multirow}
\usepackage{pdfpages}
\usepackage{placeins}
\usepackage{cellspace}
\usepackage{todonotes}
\usepackage{mdframed}
\usepackage{makecell}
\usepackage[most]{tcolorbox}

\definecolor{lightyellow}{rgb}{1.0, 0.95, 0.8}

\newcommand{\Highlight}[1]{%
  \colorbox{lightyellow}{\parbox{0.8\linewidth}{#1}}}

\lstdefinestyle{javacode}{
    language=Java,
    basicstyle=\scriptsize\ttfamily,
    numbers=none,
    escapeinside={(*@}{@*)}
}
\newtcolorbox{findingbox}[1][]{
  colback=gray!10!white,
  colframe=gray!50!black,
  #1
}


\setlength\cellspacetoplimit{3.5pt}
\setlength\cellspacebottomlimit{3.5pt}

% \def\BibTeX{{\rm B\kern-.05em{\sc i\kern-.025em b}\kern-.08em
%     T\kern-.1667em\lower.7ex\hbox{E}\kern-.125emX}}
\begin{document}

	\title {Detecting Exception-Related Behavioural Breaking Changes with UnCheckGuard}

    \author{\IEEEauthorblockN{Vinayak Sharma\IEEEauthorrefmark{1}, Patrick Lam\IEEEauthorrefmark{2}}
    \IEEEauthorblockA{\IEEEauthorrefmark{1}University of Waterloo; \emph{vinayak.sharma1@uwaterloo.ca}}
    \IEEEauthorblockA{\IEEEauthorrefmark{2}University of Waterloo; \emph{patrick.lam@uwaterloo.ca}}
    }

    \maketitle
    \thispagestyle{plain}
    \pagestyle{plain}

    \begin{abstract}
      The ubiquitous use of third-party libraries in software development has enabled developers to quickly add
      new functionality to their client software. Unfortunately, library usage also carries a cost in
      terms of software maintenance: library upgrades may include breaking changes, in which client expectations
      about library behaviour are no longer met in new library versions. Behavioural breaking
      changes can be particularly insidious, and in their full generality, could require sophisticated program
      analysis techniques to (approximately) detect.

      In this work, we present our UnCheckGuard tool, which detects a class of behavioural breaking changes---those
      related to exceptions thrown by Java libraries. UnCheckGuard analyzes both sides of the library/client
      duet. On the library side, UnCheckGuard creates a list of new exceptions that may be thrown by methods
      in a library's public API, including by its transitive callees. On the client side, UnCheckGuard identifies
      client methods that call library methods with new exceptions. To reduce false positives, UnCheckGuard
      additionally filters out new exceptions that cannot be triggered by particular clients, using taint analysis. It therefore can be
      used by client developers as a tool to screen library updates for relevant incompatibilities.

      We have evaluated UnCheckGuard on 302 libraries and 352 library-client pairs drawn from the DUETS collection
      and found 120 libraries with newly-added exceptions, as well as 1708 callsites to library methods which,
      when upgraded to the latest version, may introduce
      a behavioural breaking change in the client due to a newly added unchecked exception. These findings
      highlight the practical value of UnCheckGuard in identifying exception-related incompatibilities
      introduced by library upgrades.


    \end{abstract}
    
    

    \begin{IEEEkeywords}
        client/library interactions, behavioural breaking changes, exceptions, static analysis
    \end{IEEEkeywords}


    \section{Introduction} \label{sec:introduction}
    % look for citations about library usage
The use of libraries developed by others is ubiquitous in modern
software development. Libraries enable developers to include
functionality in their own client software without having to
implement it themselves.  However, libraries developed by others are
also updated by others, on schedules that are not controlled by the client developers.

When one is developing software that is exposed to the Internet, one
has a responsibility to incorporate at least security updates for the
libraries that one is using as a client, or else risk vulnerabilities
being exposed in one's software. The obligation to update libraries is
a form of technical debt that accrues automatically with the passage
of time.

% should we cite our Onward 2020 paper here?
However, client developers are reluctant to upgrade libraries: new
versions of libraries may include breaking Application Programming
Interface (API) changes, requiring developers to verify that their own
code continues working with the new library versions. This is
inconvenient at best.

Compilers and simple static checkers can verify the absence of
syntactic breaking changes in libraries, e.g. changes to signatures of
public methods, or retractions of formerly-existing methods. The
situation is worse for semantic breaking changes: there do not exist
techniques for reliably detecting such changes. Of course, in its full
generality, the problem is undecidable, though breaking change
detection can be estimated using program analysis techniques.

% cite FlowDroid

In this work, we contribute a way to detect one type of potentially breaking
change in a library. Our work enables client developers to identify changes
to the set of exceptions that may be thrown by a Java library, particularly
by the APIs that are actually used by client code. We leverage taint analysis
to reduce the number of false positives that we report to client developers,
and show only changed APIs that may realistically throw new exceptions
in updated versions. % something about making it easy to show test cases.

We explore the following research questions:

\noindent
{\bf RQ1.} How often do published changes to Java libraries include new added exceptions,
and under what circumstances do such exceptions occur (e.g. major/minor/patch versions)?

\noindent
{\bf RQ2.} Do library clients, in practice, call methods with new added exceptions, and is it possible for the clients to trigger these exceptions? Is it possible to write client test cases that trigger the exceptions?

\vspace*{1em}

Corpus. Summary of methodology.

The contributions of this work are as follows: \todo{clean this up}

\begin{itemize}[noitemsep]
\item empirical study of library usage
\item technique for detecting and implementation thereof
  \item results from technique
\end{itemize}

%RQ1: Do libraries have breaking changes in the form of added exceptions?
%RQ2: Do clients call the methods which have these breaking changes, and can we trigger the new exceptions?

%true, [RQ1] I think even logically it would make sense to showcase that libraries do add new unchecked exceptions (UE). We can even do something about the version in which they are adding the UE (do they introduce the UE in major version or minor version or patch version). [RQ2] Then we can follow it up with how often do clients call such method which have a newly added UE and can we trigger those UE (the taint analysis part can be defended along with it). 

% https://hasel.auckland.ac.nz/2023/11/12/understanding-breaking-changes-in-the-wild/


    \section{Background}\label{sec:background}
    We define some concepts that underpin our approach to detecting
behavioural breaking changes caused by newly added exceptions.

\textbf{Taint Analysis}. Taint analysis is a static or dynamic program analysis technique.
In taint analysis, sources (for example, client input) and sinks (for example, critical
operations or exceptions) are declared, and then it tracks whether the sources can reach the sinks.

The following example demonstrates how taint flows from a source to a sink:

\begin{lstlisting}[language=java,basicstyle=\scriptsize\ttfamily]
public class FlowDroidExampleCode {
    public static String source() {
        return "secret";
    }

    public void exampleTaintAnalysis() {
        String temp = source();
        String[] arr = new String[2];
        arr[0] = temp;
        arr[1] = "hello";
        if (arr[0] == "secret") {
            throw new RuntimeException("hello");
        }
    }
}
\end{lstlisting}

In this example, the method \texttt{source()} acts as the taint source. The statement \texttt{throw new RuntimeException("hello")} is the sink. The tainted value flows into the array \texttt{arr}, and later influences the conditional that triggers the exception. Although the exception is hardcoded, the fact that its execution depends on a tainted value makes this a valid taint flow from the source to the sink.

We apply taint analysis to detect whether newly added exceptions in a library are reachable from client-supplied values. This helps us detect behavioural breaking changes where a newly added unchecked exception is only triggered under specific conditions influenced by the client.

\textbf{SootUp}. SootUp~\cite{Karakaya24:_sootup} is a lightweight, modern Java framework built on top of Soot~\cite{vallee2010soot} that supports static analysis of Java bytecode.

SootUp transforms JVM bytecode into the intermediate representation Jimple, which simplifies analysis by converting low-level bytecode instructions into a higher-level format that captures method bodies, variable assignments, exception handling blocks, and method invocations. SootUp also provides call graph generation with various algorithms and precision levels. We use the call graph to trace the propagation of exceptions from libraries to clients. When a library method throws a new unchecked exception, we use SootUp to determine whether client methods transitively call that library method by traversing the call graph. We also use the intermediate representation to inspect methods that may throw an exception by examining throw statements and method calls within their bodies.

\textbf{FlowDroid}. FlowDroid~\cite{Arzt14:_flowdroid} is a static taint analysis framework designed for Android applications. It tracks data flow from declared sources to sinks within the application's code. Developed by Arzt et al.~\cite{Arzt14:_flowdroid}, it is built on top of the Soot~\cite{vallee2010soot} static analysis framework. FlowDroid models the complete Android lifecycle and callback structure and enables flow-sensitive, field-sensitive, context-sensitive, and object-sensitive analysis.

Even though FlowDroid primarily targets Android applications, it can still be used to perform taint analysis in general. It checks whether data from a source will taint a sink by computing possible paths along which the data can flow. In our tool, we utilize taint analysis to check the approximate reachability of newly added unchecked exceptions from client code.


    \section{Motivating Example} \label{sec:motivating}
    \input{03-motivating-example}

    \section{Data Collection}\label{sec:data-collection}
    \input{04-data-collection}


    \section{Methodology}\label{sec:methodology}
    \input{05-methodology}


    \section{Results}\label{sec:results}
    As discussed in Section~\ref{sec:data-collection}, we evaluated UnCheckGuard on 1011 Java-based clients from the DUETS dataset~\cite{durieux21:_duets}.

The goal of our tool is to detect whether a client calls a library method that, upon upgrading the library to a newer version, introduces a previously non-existent unchecked exception—potentially resulting in a behavioural breaking change.

We explore the following research questions:

\begin{itemize}
  \item[\textbf{RQ1:}] Do library clients call methods with new added exceptions, and is it possible for the clients to trigger these exceptions? Furthermore, is it possible to write client-focussed test cases that trigger the exceptions?
  \item[\textbf{RQ2:}] For library changes that introduce triggerable new unchecked exceptions, under what circumstances do such exceptions occur (i.e. major/minor/patch versions)?
\end{itemize}

Table~\ref{tab:exception-funnel} summarizes our empirical findings about the prevalence of newly-added exceptions in our corpus and how their number changes as we perform more analysis stages.

\begin{table}[h]
\centering
\caption{Exception Analysis Funnel}
\label{tab:exception-funnel}
\begin{tabular}{l r}
\toprule
\textbf{Stage} & \textbf{Count} \\
\midrule
Client invocations of external methods & 15678 \\
% Newly added exceptions called by clients & 1708 \\
% We do not have this info, it will require another run (^_^)
Exceptions passing taint analysis & 1708 \\
% Exceptions with a manually-written test case & 3 \\
% I think we should not mention this number as it is significantly small
\bottomrule
\end{tabular}
\end{table}


\subsection{Client Calls to Newly-added Exceptions}

Our evaluation includes 1011 client applications, which depend on 302 distinct libraries. Across these, we formed 352 client-library pairs in which the library had an available upgrade, each corresponding to a combination of a specific client and one of the libraries that it depends on. Table~\ref{tab:version-changes} presents the top 15 client-library pairs, ordered in descending number of callsites that pass the taint analysis reachability filter; for each pair, it also presents the number of client callsites invoking library methods with newly-added exceptions.

UnCheckGuard detected 15678 callsites across these 352 pairs where the upgraded version of the library could throw a new unchecked exception. However, it was not possible to trigger all of these exceptions using the client's methods, even with a free choice of parameters to pass to the client code. We therefore applied a taint-based reachability analysis to filter out cases that definitely could not result in actual runtime failures. After this filtering step, we identified 1708 callsites in total—spanning 120 distinct libraries—that appeared to potentially be affected by a newly added unchecked exception.

We initially tried to write test cases for those 1708 cases but were often unable to write a test case that could trigger
the newly added unchecked exception. In most of the cases, we observed that the parameters responsible for triggering the 
exceptions were not the ones passed by the client to the library method.

As with the \texttt{protobuf} case in Section~\ref{sec:methodology}, which added a new-but-untriggerable unchecked exception, taint analysis played a crucial role in reducing the number of false positives.

\vspace{1em}
\begin{tcolorbox}[colback=gray!10, colframe=black]
By adding taint analysis, we reduced the number of potentially affected callsites from 15678 to just 1708.
\end{tcolorbox}
\vspace{1em}

To assess the real-world consequences of these remaining 1708 callsites, we manually constructed test cases. For 3 of the sites, we were able to provide inputs that trigger the newly added exceptions, confirming that they represent real behavioural breaking changes.

In other cases, the exception was not triggered immediately because the client passed hardcoded values or had safeguards like null checks.

\vspace{1em}
\begin{tcolorbox}[colback=gray!10, colframe=black]
\textbf{Answer RQ1:} Yes, client applications do call methods with newly added unchecked exceptions. Out of 352 client-library pairs in our corpus, we identified 1708 callsites that reached newly-added exceptions, distributed across 136 of our 1011 clients. We were able to construct test cases that trigger the exception in some cases.
\end{tcolorbox}
\vspace{1em}

\begin{table}[h]
\centering
\caption{Distribution of reachable newly-added exceptions across version types}
\label{tab:version-distribution}
\begin{tabular}{lcc}
\toprule
\textbf{Version Type} & \textbf{Libraries} \\
\midrule
Major Version Change & 50 \\
Minor Version Change & 57 \\
Patch Version Change & 14 \\
\bottomrule
\end{tabular}
\end{table}

\begin{table*}[hbt!]
\centering
\caption{Selected clients, libraries, versions, and counts of callsites reaching newly-added exceptions}
\label{tab:version-changes}
\begin{tabular}{>{\raggedright\arraybackslash\hangindent=2em}p{3.5cm} >{\raggedright\arraybackslash\hangindent=2em}p{3.5cm} >{\raggedright\arraybackslash\hangindent=2em}p{3.5cm} >{\raggedleft\arraybackslash}p{2cm} >{\raggedleft\arraybackslash}p{2cm}}
\toprule
\textbf{Client} & \textbf{Current Version} & \textbf{Latest Version} & \textbf{Number of Callsites} & \textbf{Reachable Callsites} \\
\midrule
codes.brewing.flinkexamples-1.0-SNAPSHOT & commons-logging-1.1.1 & commons-logging-1.1.3 & 3479 & 436 \\
\addlinespace
api-2.0.2 & gson-2.3 & gson-2.13.1 & 453 & 209 \\
\addlinespace
cosyan-0.0.1-SNAPSHOT & json-20180130 & json-20250517 & 365 & 112 \\
\addlinespace
TopicModelingTool & junit-4.11 & junit-4.13.2 & 308 & 78 \\
\addlinespace
android-facebook-1.6 & android-1.6\_r2 & android-4.1.1.4 & 154 & 77 \\
\addlinespace
indextank-engine-1.0.0 & commons-cli-1.2 & commons-cli-1.10.0 & 328 & 51 \\
\addlinespace
commons-pipeline-1.0-SNAPSHOT & commons-digester-1.7 & commons-digester-2.1 & 76 & 48 \\
\addlinespace
codes.brewing.flinkexamples-1.0-SNAPSHOT & commons-codec-1.3 & commons-codec-1.4 & 47 & 38 \\
\addlinespace
mrdpatterns-1.0-SNAPSHOT & hadoop-core-1.1.1 & hadoop-core-1.2.1 & 466 & 36 \\
\addlinespace
rehttp & xembly-0.31.1 & xembly-0.32.2 & 61 & 36 \\
\addlinespace
indextank-engine-1.0.0 & log4j-1.2.16 & log4j-1.2.17 & 33 & 33 \\
\addlinespace
Timeline-2.0.0 & tablestore-4.11.2 & tablestore-5.17.6 & 479 & 32 \\
\addlinespace
HospitalAction-1.0 & poi-5.2.2 & poi-5.4.1 & 42 & 28 \\
\addlinespace
MavenProject-0.0.1-SNAPSHOT & selenium-api-3.141.59 & selenium-api-4.35.0 & 120 & 24 \\
\addlinespace
amazon-kinesis-aggregators-.9.2.9 & commons-logging-1.1.1 & commons-logging-1.3.5 & 105 & 23 \\
\bottomrule
\end{tabular}
\end{table*}

\subsection{Newly-added Unchecked Exceptions in Java Libraries}
Semantic versioning~\cite{preston-werner23:_seman_version} proposes that version numbers have three parts, $x.y.z$. According to semantic versioning, library developers are to change the major version $x$ when an upgrade is breaking---that is, a client may have to modify their code to use the new versioning. Minor version upgrades (indicated by changes to $y$) may include new features, while patch upgrades (changes to $z$) fix bugs.

Table~\ref{tab:version-distribution} shows the distribution of newly-added exceptions reachable from clients, across upgrade types. Notably, 50 out of these 120 libraries introduced new unchecked exceptions as part of a major version bump. However, we also observed 14 cases in a patch version upgrade. While we are not making any broader claims about how often behavioural breaking changes occur in general, our results indicate that minor and patch upgrades do introduce behavioural breaking changes via unchecked exceptions which may affect clients—something that developers may not anticipate.

\vspace{1em}
\begin{tcolorbox}[colback=gray!10, colframe=black]
\textbf{Answer RQ2:} Java libraries introduce newly added unchecked client-relevant exceptions across versions frequently enough to be relevant to clients. We found newly added unchecked exceptions in 120 out of 302 distinct libraries (39.7\%). These changes in major version upgrades (50 times), minor version upgrades (57 times), and patch (14 times) version upgrades (e.g., \texttt{httpcore-4.4.6}~$\rightarrow$~\texttt{httpcore-4.4.16}).
\end{tcolorbox}
\vspace{1em}

\subsection{Discussion: Developer-Facing Implications}

Behavioural breaking changes caused by unchecked exceptions during API evolution are particularly dangerous. Such changes do not show up at compile time, and they do not affect method signatures, which means that the existing tools that we are aware of cannot detect them. For instance, both japicmp and Revapi, widely used tools for detecting breaking changes, focus on syntactic differences in method signatures. While they can both flag checked exceptions—since they appear in method declarations—they do not analyze the method implementations, and thus have no way of identifying newly added unchecked exceptions. As a result, developers who rely solely on either japicmp or revapi could remain unaware of serious runtime-breaking issues.

Some tools have tried to tackle the challenge of behavioural breaking changes. CompCheck~\cite{CompCheck}, for example, works by identifying test cases in some clients and reusing them for others with similar API usage. But this approach depends heavily on the presence of thorough test suites. Most clients that we have looked at do not have such comprehensive coverage, especially not for edge cases involving unchecked exceptions.

This is where UnCheckGuard steps in. Unlike existing work, it does not rely on existing test cases. Instead, it compares the old and new versions of a library using static analysis to detect newly added unchecked exceptions, and then runs taint analysis to filter out changes that do not affect the client. By avoiding the need for a test suite, it can reveal behavioural breaking changes that other tools overlook.

In doing so, UnCheckGuard addresses an important gap. It gives developers visibility into a class of breaking changes that are easy to miss but costly in practice—helping them catch potential failures early, before they reach production.



    % TODO: add discussion
    % TODO: add threats to validity

    %\section{Threats to Validity}\label{sec:threats}
    %\input{threats}
    
    \section{Related Work}\label{sec:related-work}
    \input{07-related-work}
    

    \section{Conclusion}\label{sec:conclusion-&-future-work}
    \input{08-conclusion}
    
    \bibliography{references}

\end{document}
